\documentclass{article}
\usepackage{mathpartir}
\begin{document}

\section{Language of Types}

The language of types is described by the following grammar, where $A$ ranges over qualifier literals:

\begin{math}
\begin{array}{rcl}
\sigma & ::= & \forall \alpha . \sigma\ |\ \omega \\

\omega & ::= & \Lambda \kappa \le Q . \omega\ |\ \gamma \\

Q & ::= & \top\ |\ \bot\ |\ A \\

\gamma & ::= & \rho \eta \\

\eta & ::= & \gamma \rightarrow \gamma\ |\ \tau \\

\rho & ::= & Q\ |\ \kappa\ |\ \rho \sqcup \rho\ |\ \rho \sqcap \rho \\

\tau & ::= & \alpha\ |\ \textrm{int}\ |\ \textrm{bool} \\

\end{array}
\end{math}

We also restrict arrow types so that the first occurrence of a qualifier variable must be outside of a union, e.g., types like ${\kappa}_1 \sqcup {\kappa}_2 \tau \rightarrow {\kappa}_1 \tau$ and ${\kappa}_1 \rightarrow {\kappa}_1 \sqcup {\kappa}_2 \tau \rightarrow {\kappa}_2 \tau$ are not permitted, but types like ${\kappa}_1 \tau \rightarrow {\kappa}_2 \tau \rightarrow {\kappa}_1 \sqcup {\kappa}_2 \tau$ are.
  The same restriction applies to intersection types.


\section{Expression Language}

\begin{math}
\begin{array}{rrl}

e & ::= & n \\
  & | & x \\
  & | & \textrm{true} \\
  & | & \textrm{false} \\
  & | & \textrm{if } e \textrm{ then } e \textrm{ else } e \\
  & | & \textrm{let } x:\sigma = e \textrm{ in } e \\
  & | & \textrm{fun } x:\gamma = e \\
  & | & e e \\
  & | & [ A ] e \\

\end{array}
\end{math}


\section{Type Checking}

The typing checking rules for System WTF are as follows, with each metavariable ranging over the types constructable from its production in the preceding grammar:

\subsection{Subtype Relation}

\begin{mathpar}

\inferrule[sub-ty]
  {\rho' \sqsubseteq \rho}
  {\rho' \tau \le \rho \tau}

\inferrule[sub-func]
  {\rho' \sqsubseteq \rho \\ {\gamma}_1' \le {\gamma}_1 \\ {\gamma}_2 \le {\gamma}_2'}
  {\rho' ({\gamma}_1' \rightarrow {\gamma}_2') \le \rho ({\gamma}_1 \rightarrow {\gamma}_2)}

\inferrule[sub-tyabs]
  {\sigma' \le \sigma}
  {\forall \alpha . \sigma' \le \forall \alpha . \sigma}

\inferrule[sub-qualabs]
  {Q \sqsubseteq Q' \\ \omega'[\kappa \mapsto Q] \le \omega[\kappa \mapsto Q] }
  {\Lambda \kappa \sqsubseteq Q' . \omega' \le \Lambda \kappa \sqsubseteq Q . \omega}

\end{mathpar}


\subsection{Annotation Rules}

The following rules are used to formalize the notion that qualifier annotation occurs underneath type and qualifier abstractions.

\begin{mathpar}

\inferrule[annot]
  { }
  {A \sqcap \rho \eta \hookrightarrow (A \sqcap \rho) \eta}

\inferrule[annot-tyabs]
  {A \sqcap \sigma \hookrightarrow \sigma'}
  {A \sqcap \forall \alpha . \sigma \hookrightarrow \forall \alpha . \sigma'}

\inferrule[annot-qualabs]
  {A \sqcap \omega \hookrightarrow \omega'}
  {A \sqcap \Lambda \kappa \le Q . \omega \hookrightarrow \Lambda \kappa \le Q . \omega'}

\end{mathpar}


\subsection{Typing Rules}

\begin{mathpar}

\inferrule[int]
  { }
  {\Gamma \vdash n : \top \textrm{ int}}

\inferrule[true]
  { }
  {\Gamma \vdash \textrm{true} : \top \textrm{ bool}}

\inferrule[false]
  { }
  {\Gamma \vdash \textrm{false} : \top \textrm{ bool}}

\inferrule[var]
  { }
  {\Gamma \vdash x : \Gamma(x)}

\end{mathpar}


\begin{mathpar}

\inferrule[annot]
  {\Gamma \vdash e : \sigma \\ \Gamma \vdash A \sqcap \sigma \hookrightarrow \sigma'}
  {\Gamma \vdash [A] e : \sigma'}

\end{mathpar}


\begin{mathpar}

\inferrule[if]
  {\Gamma \vdash e_1 : \top \textrm{ bool} \\ \Gamma \vdash e_2 : {\sigma}_2 \\ \Gamma \vdash e_3 : {\sigma}_3 \\ {\sigma}_2 \le \sigma \\ {\sigma}_3 \le \sigma}
  {\Gamma \vdash \textrm{if } e_1 \textrm{ then } e_2 \textrm{ else } e_3 : \sigma}

\end{mathpar}


\begin{mathpar}

\inferrule[let]
  {\Gamma \vdash e' : \sigma' \\ \Gamma, x:\sigma \vdash e : \gamma \\ \sigma' \le \sigma}
  {\Gamma \vdash \textrm{let } x : \sigma = e' \textrm{ in } e : \gamma}

\end{mathpar}


\begin{mathpar}

\inferrule[abs]
  {\Gamma,x:\gamma \vdash e : \gamma'}
  {\Gamma \vdash \textrm{fun } x : \gamma = e : \gamma \rightarrow \gamma'}

\inferrule[app]
  {\Gamma \vdash e_1 : \rho (\gamma \rightarrow \gamma') \\ \Gamma \vdash e_2 : \gamma_2 \\ {\gamma}_2 \le \gamma}
  {\Gamma \vdash e_1 e_2 : \gamma'}

\end{mathpar}


\begin{mathpar}

\inferrule[tyabs]
  {\Gamma \vdash e : \sigma}
  {\Gamma \vdash \forall \alpha . e : \forall \alpha . \sigma}

\inferrule[tyapp]
  {\Gamma \vdash e : \forall \alpha . \sigma}
  {\Gamma \vdash e[\tau] : \sigma[\alpha \mapsto \tau]}

\end{mathpar}


\begin{mathpar}

\inferrule[qualabs]
  {\Gamma \vdash e : \omega}
  {\Gamma \vdash \Lambda \kappa \sqsubseteq Q . e : \Lambda \kappa \sqsubseteq Q . \omega}

\inferrule[qualapp]
  {\Gamma \vdash e : \Lambda \kappa \sqsubseteq Q . \omega \\ A \sqsubseteq Q}
  {\Gamma \vdash e\{A\} : \omega[\kappa \mapsto A]}

\end{mathpar}

Note that we've changed our notion of qualifier application so that instantiation is final, i.e., exactly like in type application, where before there was some notion of there being several instantiations of a qualifier variable.

\end{document}
